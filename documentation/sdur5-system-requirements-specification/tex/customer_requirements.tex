% Include a header paragraph specific to your product here. Customer requirements are those required features and functions specified for and by the intended audience for this product. This section establishes, clearly and concisely, the "look and feel" of the product, what each potential end-user should expect the product do and/or not do. Each requirement specified in this section is associated with a specific customer need that will be satisfied. In general Customer Requirements are the directly observable features and functions of the product that will be encountered by its users. Requirements specified in this section are created with, and must not be changed without, specific agreement of the intended customer/user/sponsor.

% \subsection{Requirement Name}
% \subsubsection{Description}
% A detailed description of the feature/function that satisfies the requirement. For example: \textit{The GUI background will be slate blue. This specific color is required in order to ensure that the GUI matches other similar software products offered by the customer. Slate blue is specified as \#007FFF, using six-digit hexadecimal color specification.} It is acceptable and advisable to include drawings/graphics in the description if it aids understanding of the requirement.
% \subsubsection{Source}
% The source of the requirement (e.g. customer, sponsor, specified team member (by name), federal regulation, local laws, CSE Senior Design project specifications, etc.)
% \subsubsection{Constraints}
% A detailed description of realistic constraints relevant to this requirement. Economic, environmental, social, political, ethical, health \& safety, manufacturability, and sustainability should be discussed as appropriate.
% \subsubsection{Standards}
% A detailed description of any specific standards that apply to this requirement (e.g. \textit{NSTM standard xx.xxx.x. color specifications \cite{Rubin2012}}. Standards exist for practically everything (ATC standard fuses, IEEE 802.15.4 embedded wireless, TLS 1.3 encryption, etc.), so be sure that you research and document which ones will be followed in meeting this requirement.
% \subsubsection{Priority}
% The priority of this requirement relative to other specified requirements. Use the following priorities:
% \begin{itemize}
% \item Critical (must have or product is a failure)
% \item High (very important to customer acceptance, desirability)
% \item Moderate (should have for proper product functionality);
% \item Low (nice to have, will include if time/resource permits)
% \item Future (not feasible in this version of the product, but should be considered for a future release).
% \end{itemize}

% \subsection{Requirement Name}
% \subsubsection{Description}
% Detailed requirement description...
% \subsubsection{Source}
% Source
% \subsubsection{Constraints}
% Detailed description of applicable constraints...
% \subsubsection{Standards}
% List of applicable standards
% \subsubsection{Priority}
% Priority

The customers of the UR5 Checkers-Playing Co-bot include, but are not limited to, the project sponsor, the developers, peers, UTA College of Engineering, and prospective students. This section establishes what the end-user should expect with the product. The co-bot will come with a custom-made checkerboard and custom-made checkers pieces to ensure that the physical components of the game are large enough to be detected by the UR5 co-bot's magnetic electro-permanent gripper. The product will also include a collection box for both the user and robot to deposit the checkers pieces during the match. The user will have control over terminating or resetting the match.   

\subsection{The product shall have a custom-made checkerboard.}
\subsubsection{Description}
The checkerboard will be constructed of a 18 inch by 18 inch wooden board and be custom-painted. It is custom-made to ensure that the size of the checkerboard and checkers pieces are large enough to be easily detected and grasped by the robot arm.
\subsubsection{Source}
Kevin Vu
\subsubsection{Constraints}
The custom board must be small enough to comfortably fit the UR5 stand base, and the squares must be large enough to fit the checkers pieces.
\subsubsection{Standards}
N/A
\subsubsection{Priority}
Critical

\subsection{The product shall have custom-made checker pieces.}
\subsubsection{Description}
These checker pieces will be 3D printed and will have a small magnet contained within them for the robot to grip the piece.
\subsubsection{Source}
Hoang Ho
\subsubsection{Constraints}
The checker pieces need to have a hole cut out for the washers to be inserted in them for easy pickup by the magnet in the robot arm. 
\subsubsection{Standards}
N/A
\subsubsection{Priority}
Critical

\subsection{The product shall have a piece collection box.}
\subsubsection{Description}
The box will hold the checkers pieces that the robot arm picks up throughout a checkers match. Additionally, when storing away the game, the checkers pieces can go in here.
\subsubsection{Source}
Kevin Vu
\subsubsection{Constraints}
The box must be in reach of the UR5 robot arm during game play to drop off its checker pieces.
\subsubsection{Standards}
N/A
\subsubsection{Priority}
High

\subsection{The UR5 robot arm shall have a magnetic electro-permanent gripper.}
\subsubsection{Description}
This gripper will have the ability to turn on to grip, and turn off to ungrip. The gripper will be used by the robot arm to pick up checker pieces, and make its move.
\subsubsection{Source}
Hoang Ho
\subsubsection{Constraints}
Objects may need to have certain magnetic strength to be able to be picked up by the gripper. On the other hand, the gripper magnetic strength may be needed to take into account also, so that checker pieces relative to the one the robot arm is picking up is not picked up either. It may be that we may need to use a different type of gripper for the optimal object pick up motion. We may be limited by the different magnets we can buy whether it be size, strength, or budget. These limitations also apply to the gripper itself.
\subsubsection{Standards}
N/A
\subsubsection{Priority}
Critical

\subsection{The system shall follow the rules of American checkers.}
\subsection{Description}
The checkers engine should be programmed to know how to play by the rules of American checkers (English draughts) to avoid confusion.
\subsection{Source}
Kevin Vu
\subsection{Constraints}
N/A
\subsection{Standards}
N/A
\subsection{Priority}
High

\subsection{The system shall be able to abort and reset mid-match.}
\subsubsection{Description}
For demonstration purposes, the checkers program should be able to abort mid-match to prepare for the next round of tourists/viewers to play or watch a fresh match.
\subsubsection{Source}
Kevin Vu
\subsubsection{Constraints}
N/A
\subsubsection{Standards}
N/A
\subsubsection{Priority}
High

\subsection{The product shall show the state of the game to the player at all times.}
\subsubsection{Description}
This will be used to provide a computerized version of the game.
\subsubsection{Source}
Nimita Uprety
\subsubsection{Constraints}
N/A
\subsubsection{Standards}
N/A
\subsubsection{Priority}
High