% What lab space, testing grounds, makerspaces, etc. will you need to complete the project? Will you require any specific equipment, and if so, where will you get it (borrow, lease, purchase, outsource, already present in the lab, etc.). This section should occupy 1/2 page.

For our project we expect to be primarily utilizing the senior design lab facilities located in ERB 335 as this is where the UR5 Robot Arm should be stored throughout the majority of the year. This space will also be utilized by our team for the development, testing and implementation of any components we add to our project such as the camera and checker piece gripper. That said, the co-bot is currently being used on other projects and is not present in the senior design lab at the moment. The UR5 arm is expected to return to the ERB 335 lab soon but its final location is subject to change.

Given that our team is still undecided on what kind of gripper we will be implementing to pick up our checkers pieces we may or may not have an additional facility that we use. This is due to the fact that we are currently exploring two options, a vacuum gripper and an electropermanent magnetic one. If our team decides to make the gripper accessory magnetic, this means we will also need to 3D print our own checkers pieces so that we can put magnets inside them. In the event that we choose this option then we will utilize the available 3D printers in the FabLab to build our custom game pieces.

In terms of equipment the most vital piece is the UR5 Robot Arm itself. This piece will be leased from the College of Engineering at the University of Texas at Arlington. Additional equipment items include 3D printers (located in the FabLab) \& design software, Robot Operating Software [ROS]. However due to the complexity of our project this is not a final list of equipment pieces as we may need to use additional items to create our gripper accessories \& checkers pieces.
