The UR5 collaborative robot (co-bot) belonging to the Computer Science department at the University of Texas at Arlington currently does not showcase the collaborative nature of its ability to interact closely with humans. The UR5 co-bot will be programmed to perform a complete game of the strategy board game, checkers, against a human opponent. The purpose of the UR5 co-bot project is to use the programmed co-bot as a marketing strategy for UT Arlington to expose students to industrial automation technology, introduce students to the UT Arlington College of Engineering and Computer Science Engineering departments, and provide an enjoyable educational experience. 

%The problem statement defines the "Why" of the project. This is the higher purpose, or the reason for the project's existence. This section should avoid mentioning implementation details, and focus more on what the current problem is and what would be gained if the problem were to be solved. In short, the is the reason that you are going to be working on something, not the method(s) that you will be employing.