% An in-depth explanation of the problem, including the "business case". What is wrong with the status-quo or what opportunity exists that justifies undertaking this project (expanding upon the problem statement)? If you have a clear customer or sponsor, why do they want you to work on this? What is the existing relationship, if any, between the development team and the customer? This section should occupy 1/2 - 1 full page.

Today, we live in an age of rapid technological advancements and increasing usages of mechanical and digital automation. Time and time again, modern technology has continued to showcase how it can be utilized to enrich the human experience. However, in order to continue advancing our technologies, we must pique the interest and inspire a new generation of innovators and engineers. In order to inspire this new generation, we can show them how we can apply new modern technologies to enrich existing familiar experiences.

Our project aims to showcase how a robot arm can be programmed to play a game of checkers, a game very familiar to many. Although there is no real need or demand for a checkers-playing robot today, there is a very real and growing demand for engineers in the modern world, especially when it comes to computing technologies. With this project, we hope to ignite new and existing passions for technology and innovation.

This isn't to say something similar to this hasn't been done before. There have been many projects in the past that involve a robot completing familiar tasks. Whether it be cooking dishes, brewing coffee, solving Rubik's cubes, or playing other board games, they all still have a similar aim in mind: to showcase the vast capabilities of new and rising technologies and give these technologies much needed exposure. We hope to follow in their footsteps and showcase the collaborative capabilities of the UR5 collaborative robot.

Through the use of computer vision and the UR5 collaborative robot arm, we aim to have the robot arm be able to play a game of checkers against another player, not just against itself or another robot. We intend to showcase the UR5 robot arm's ability to safely interact closely with humans.

Our customer, Dr. Christopher McMurrough, has expressed that he would be happy to have more cool and exciting projects to showcase to prospective students that visit the UT Arlington campus in order to inspire and encourage them to pursue further education in the engineering field. We hope this project can draw the attention of future UT Arlington students and show them what can be done with computing technology.